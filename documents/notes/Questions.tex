\section{Questions that arise from this situation}
\label{sec:ai}





   This situation raises questions about how AI is using the information we put on the internet. As of now there is very little legislation stating what AI can and cannot do. AI is progressing faster than we can regulate it, and that is a problem. Recently it has become known that Google and OpenAI have started using transcribed videos to train their AI’s. This raises questions about copyright infringement. Any video that you put on the internet could eventually be used for training and AI programs, and you wouldn’t even know it. As of now there is no way to know what specific things an AI like ChatGPT has been trained on.
   
   This situation also raises questions about the imitation of peoples voices. Many funny AI generated videos of Joe Biden, Donald Trump, and many other people have been posted online for millions to see. This raises the question: do we have rights over our own voice? As of now the answer seems to be no. Anyone with the appropriate software can imitate your voice as long as they have enough data to train the AI. 
   This issue will only become larger as AI impersonation gets better and requires less training data on someone's voice. 
