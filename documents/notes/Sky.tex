\section{Scarlett Johanson and the voice of Sky}
\label{sec:ai}




  
Scarlett Johanson was contacted last September by Open AI to become a voice for the new ChatGPT version. She declined due to personal reasons. Notably, Scarlett Johanson said this while describing the phone call with this representative. "He told me that he felt that by voicing the system, I could bridge the gap between tech companies and creatives and help consumers to feel comfortable with the seismic shift concerning humans and AI. He said he felt that my voice would be comforting to people." This shows that one of OpenAI’s goals is to make people feel as comfortable as possible while using the app, and it seems they are doing this by making the AI feel as human as possible. 

After this phone call Scarlett Johanson wasn’t in notable contact with OpenAI until two days before the ChatGPT 4.0 demo was released. OpenAI contacted her asking if she would reconsider. Scarlet Johnson declined and two days later the demo was released. This demo contained a voice called “sky”, which had remarkable resemblance to the AI from “her”, and as such, Scarlett Johansson's voice. Along with this, the CEO of open AI tweeted “her” on his X account before the demo went live. OpenAI denies any allegations of using Scarlett Johansons voice for the voice of Sky, however they did end up removing the Sky voice after a few days of it being live. 

Scarlett Johanson had this to say about the situation: 

“When I heard the released demo, I was shocked, angered and in disbelief that Mr. Altman (CEO of OpenAI) would pursue a voice that sounded so eerily similar to mine that my closest friends and news outlets could not tell the difference,”

Because of this situation Scarlett Johanson is seeking legal counsel and asking OpenAI to detail the exact process they used to train Sky. \cite{scarlett-johansson-article1} \cite{scarlett-johansson-article2}
